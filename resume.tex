%#!platex resume.tex; dvipdfmx resume.dvi
\documentclass{resume}
%\usepackage{amssymb,amsmath}
%\usepackage{listings,jvlisting}
%\usepackage{multicol}
%\usepackage[hyphens]{url}
%\usepackage{graphicx}
%\usepackage[dvipdfmx]{graphicx}
%\usepackage[dvipdfmx]{color}
%\usepackage{xcolor}
%\usepackage{here}
%\usepackage{mathtools}
%\usepackage{stmaryrd}
%\usepackage[dvipsnames]{xcolor}
\usepackage{graphicx}
%\usepackage[dvipdfmx]{graphicx}
\usepackage[dvipdfmx]{color}
\usepackage{float}
\usepackage{listings}
\usepackage{jvlisting} %日本語のコメントアウトをする場合jvlisting(もしくはjlisting)が必要
\usepackage{amsmath}
\usepackage{mathtools}
\usepackage{stmaryrd} % \llparenthesis & rrparenthesis
\usepackage[dvipsnames]{xcolor}
\usepackage[hyphens]{url}
\lstset{
  basicstyle={\ttfamily},
  identifierstyle={\small},
  commentstyle={\small\itshape},
  keywordstyle={\small\bfseries\color{Green}}, % キーワードの色を青に設定
  ndkeywordstyle={\color{orange}},
  stringstyle={\small\ttfamily\color{orange}}, % 文字列の色を緑に設定
  frame={tb},
  breaklines=true,
  columns=[l]{fullflexible},
  numbers=left,
  xrightmargin=0zw,
  xleftmargin=0zw,
  numberstyle={\scriptsize},
  stepnumber=1,
  numbersep=0.5zw,
  lineskip=-1.0ex,
  captionpos=b,
  keywords={At}, 
  ndkeywords={com,select},% 追加のキーワードを指定
  moredelim=**[is][\color{gray}]{!}{!}, % 特定の要素を赤に設定
  moredelim=**[is][\color{red}]{|}{|},
  moredelim=**[is][\color{blue}]{`}{`},
}
%%%%%%%%%%%%%%%%%%%%%%%%%%%%%%%%%%%%%%%%%%%%%%%%%%
%%% ヘッダ設定
\所属{岐阜大学大学院 自然科学技術研究科 知能理工学専攻}
\研究室{草刈研究室}
\氏名{恩田晴登}
\題目{Pythonの静的型検査器を活用した\\コレオグラフィックプログラミング言語の実装}

%%%%%%%%%%%%%%%%%%%%%%%%%%%%%%%%%%%%%%%%%%%%%%%%%%
%%% ユーザ定義
\renewcommand{\lstlistingname}{Code}
\newcommand{\projection}[2]{{\color{cyan}\llparenthesis}#1{\color{cyan}\rrparenthesis^#2}}
\newcommand{\mblue}[1]{\textbf{\textsf{\color{MidnightBlue}#1}}}
\newcommand{\cyan}[1]{\color{cyan}#1}
\newcommand{\gr}[1]{\color{ForestGreen}#1}
\newcommand{\nl}[1]{{\color{red}{\llbracket}}#1{\color{red}{\rrbracket}}} % Normalizer symbol
\newcommand{\mg}{~{\color{red}{\sqcup}}~} % Merging symbol

%%%%%%%%%%%%%%%%%%%%%%%%%%%%%%%%%%%%%%%%%%%%%%%%%%
%%% 本文
\begin{document}
\maketitle
\bibliographystyle{jplain}
%%%%%%%%%%%%%%%%%%%%%%%%%%%%%%%%%%%%%%%%%%%%%%%%%%
\section{はじめに}

%機械学習やIoTの分野で盛んに使用されているPythonは並行,並列,および分散プログラミングの需要がある.しかし,
一般的に,並行・分散プログラムは,デッドロック等の並行性に起因するエラーや非決定性問題の
発見と修正が困難であるため構築が難しい.この問題の解決手法の一つとして,コレオグラフィが提唱されている.
コレオグラフィは並行に動作する複数参加者の連携手順をまとめたプログラムであり,コレオグラフィックプログラミング言語によって記述される.
コレオグラフィに準ずる各参加者のプログラムは,\textbf{エンドポイント射影}と呼ばれる操作によって自動的に生成される.
%コレオグラフィに従った各参加者のプログラムは,エンドポイント射影によって自動的に生成されるメカニズムが備わっている.
%エンドポイント射影とは,コレオグラフィから各参加者のプログラムを導出する操作である.
生成された各参加者のプログラムはデッドロック等の並行性に起因するエラーがないことが理論的に保証されている.
先行研究であるChoral \cite{choral}は,Javaを拡張したコレオグラフィックプログラミング言語の一つである.
しかし,機械学習やIoTの分野で盛んに使用されているPythonを拡張したコレオグラフィックプログラミング言語は
著者が知る限りはまだ提案されていない.

本研究では,Pythonをベースとしたコレオグラフィックプログラミング言語\textbf{PyChoral}を構築し,有用性を確かめる.
これにより,機械学習やIoTを扱うプロジェクトにおいて、Pythonで並行・分散プログラムが実装可能となることを目指す足がかりとする.
本研究の特徴はPythonの型検査器である\textbf{Mypy}の型検査結果を活用することである.
PyChoralの構文はPythonと同一であることから,PythonのIDEやライブラリを使うことができる.
よって,Pythonの標準的な仕組みを保ったまま,単一のプログラムで並行・分散プログラムを記述できるという点で可用性をもつ.


\section{PyChoralによるプログラミング}

PyChoralはPythonの構文や型システムを踏襲して構築されている.PyChoralを使ったコレオグラフィはクラスを用いて構築する.
参加者間では,コレオグラフィの中で生成するチャネルを用いて通信する.
Code \ref{ch}は,入力した整数が2で割り切れる回数をカウントするコレオグラフィ\textsf{Div2}である.
\texttt{A}と\texttt{B}は\textsf{div2}メソッドを呼ぶ.
\texttt{A}に引数で渡される整数値\texttt{n}が2で割り切れる限り,\texttt{n}を2で割って,割った回数をカウントする.
もし割り切れない場合は,それまでにカウントした回数を\texttt{A}から\texttt{B}へ送る.例えば,ユーザーコード側で\texttt{A}のメソッド\textsf{div2}に$(8,0)$を渡すと\texttt{B}は3を受け取る.
%\texttt{A}に引数で渡される整数値\texttt{n}が2で割り切れれば,
%メソッド\textsf{div2}に\texttt{n}を2で割った値と,割る毎に1カウントアップする値\texttt{c}を引数として再帰呼び出しする.
%割り切れなくなったらそれまでにカウントした回数を戻り値として返す.例えば,ユーザーコード側で\texttt{A}のメソッド\textsf{div2}に$(8,0)$を渡すと3が返ってくる.
2,3行目はクラスのコンストラクタを表す.ここでは\texttt{A}から\texttt{B}へobject型の値を送るチャネル\texttt{chAB}を生成している.

4-10行目は\texttt{A}と\texttt{B}が連携して動作するメソッド\texttt{div2}である.
5行目のif文で,\texttt{A}がもつ整数\texttt{n}が2で割り切れるかどうかで分岐する.この分岐の結果はまだ\texttt{B}には知らされない.
6,7行目は割り切れる場合,9,10行目は割り切れない場合である.
6,9行目にある\texttt{select}は条件分岐の結果を示す列挙型の値を送るメソッドである.
\texttt{com}は送信者の情報を含む任意の型をもつ値を送るメソッドである.
\begin{lstlisting}[caption=PyChoralによるコレオグラフィ,label=ch]
class Div2(Choreograhy2[A,B]):
  def __init__(self) -> None:
    self.chAB = Channel[object,A,B]()
  def div2(self,n:At[int,A],c:At[int,A]) -> At[int,B]:
    if (n % 2)@A() == 0@A():
      self.chAB.select(|OddEven.EVEN|@A())
      return self.div2((n//2)@A(),(c+1)@A())
    else:
      self.chAB.select(`OddEven.ODD`@A())
      return self.chAB.com(c@A())
\end{lstlisting}
\vspace*{-7pt}

クラス\texttt{Div2(Choreograhy2[A,B])}からエンドポイント射影により,クラス\texttt{Div2\_A},\texttt{Div2\_B}が生成される.
参加者\texttt{A,B}はプロセスから\texttt{Div2\_A},\texttt{Div2\_B}のメソッド\texttt{div2}を呼び出すことにより連携して動作する.
Code \ref{python}は,Code \ref{ch}から射影された\texttt{B}のPythonプログラムである.型注釈や\texttt{B}が関連しない式や文は除去される.
\begin{lstlisting}[caption=参加者\texttt{B}のPythonプログラム,label=python]
class Div2_B():
  def __init__(self):
    self.chAB = Channel[object,A,B]()
  def div2(self):
    match self.chAB.select():
      case |OddEven.EVEN|:
        return self.div2()
      case `OddEven.ODD`:
        return self.chAB.com()
      case _:
\end{lstlisting}
\vspace*{-7pt}

分岐を含むコレオグラフィのエンドポイント射影で重要なのは,ある参加者で発生した条件分岐をいかにして他の参加者に伝えるか,という仕組みを実現することである.これを実現するのが\textbf{マージ}である.
PyChoralにおけるエンドポイント射影およびマージの理論はChoralを模倣している.
if文におけるマージは,if文の条件式を判断した参加者以外が条件式の結果に相当する列挙型の値を受信して分岐することにより実現する.
PyChoralではPython3.10から導入されたmatch文を使用している.
match文のケースはCode \ref{ch}の6,9行目にある\texttt{select}メソッドで送信した列挙型の値で分ける.
Code \ref{ch}の6行目から射影によりCode \ref{python}の5-7,10行目が生成される.
Code \ref{ch}の9行目からは射影によりCode \ref{python}の5,8-10行目が生成される.
\texttt{case \_} は共通のケースであるため,統合されて一つになる.
\texttt{{\color{red}{OddEven.EVEN}},{\color{blue}{OddEven.ODD}}}は独立したケースであるため,マージ後のmatch文にそのまま加える.
これにより,条件分岐に関係ない参加者も分岐を考慮した形でエンドポイント射影される.
生成された\texttt{A,B}のPythonプログラムは,エンドポイント射影およびマージによりデッドロックが起こることなく連携して動作することが可能である.

%PyChoralはPythonの構文や型システムを踏襲して構築されている.PyChoralを使ったコレオグラフィはクラスを用いて構築する.
%参加者間では,コレオグラフィ中で生成するチャネルを用いて通信する.
%Code \ref{ch}は参加者\texttt{A}と\texttt{B}との挨拶のコレオグラフィである.
%\texttt{A}と\texttt{B}は\textsf{greeting}メソッドを呼ぶ.\texttt{A}は引数としてメッセージ\texttt{msg}を渡し,戻り値として\texttt{B}からのメッセージを受け取る.
%\texttt{B}には引数,戻り値ともにない.\texttt{A}が渡した文字列が\texttt{"hello"}であれば,\texttt{B}から送られてくる戻り値が\texttt{"hello"}となる.
%それ以外ならば,\texttt{"bye"}となる.
%2-4行目はクラスのコンストラクタを表す.ここでは\texttt{A}から\texttt{B}へ列挙型の\texttt{Choice}を送るチャネル\texttt{chAB}と,\texttt{B}から\texttt{A}へ文字列を送るチャネル\texttt{chBA}を生成している.
%5-11行目は\texttt{A}と\texttt{B}が連携して動作するメソッド\texttt{greeting}である.
%\texttt{A}から\texttt{B}に送るメッセージ\texttt{msg}を引数にとり,それに基づいて動作する.
%6行目のif文で,\texttt{A}が\texttt{B}に送るメッセージが\texttt{"hello"}であるかどうかで分岐する.この分岐の結果はまだ\texttt{B}にはまだ知らされない.
%7,8行目は\texttt{"hello"}を送る場合,10,11行目は\texttt{"hello"}以外を送る場合である.それぞれ\texttt{B}から\texttt{A}に\texttt{"hello"}か\texttt{"bye"}を送る.
%7,10行目にある\texttt{select}は条件分岐の結果を示す列挙型の値を送るメソッドである.
%\texttt{com}は送信者の情報を含む任意の型をもつ値を送るメソッドである.
%\begin{lstlisting}[caption=参加者\texttt{A,B}のコレオグラフィ,label=ch]
%  class Greeting(Choreograhy2[A,B]):
%    def __init__(self) -> None:
%      self.chAB = Channel[Choice,A,B]()
%      self.chBA = Channel[str,B,A]()
%    def greeting(self,msg:At[str,A]) -> At[str,A]:
%      if msg == "hello"@A():
%        self.chAB.select(|Choice.HELLO|@A())
%        return self.chBA.com("hello"@B())
%      else:
%        self.chAB.select(`Choice.BYE`@A())
%        return self.chBA.com("bye"@B())
%  \end{lstlisting}
%  \vspace*{-7pt}

%クラス\texttt{Greeting(Choreograhy2[A,B])}からエンドポイント射影により,クラス\texttt{Greeting\_A},\texttt{Greeting\_B}が生成される.
%参加者\texttt{A,B}はプロセスから\texttt{Greeting\_A},\texttt{Greeting\_B}のメソッド\texttt{greeting}を呼び出すことにより連携して動作する.
%Code \ref{python}はCode \ref{ch}から射影された\texttt{B}のPythonプログラムである.型注釈や\texttt{B}が関連しない式や文は除去される.
%\begin{lstlisting}[caption=参加者\texttt{B}のPythonプログラム,label=python]
%class Greeting_B():
%  def __init__(self):
%    self.chAB = Channel[Choice,A,B]()
%    self.chBA = Channel[str,B,A]()
%  def greeting(self):
%    match self.chAB.select():
%      case |Choice.HELLO|:
%        return self.chBA.com("hello")
%      case `Choice.BYE`:
%        return self.chBA.com("bye")
%      case _:
%\end{lstlisting}
%\vspace*{-7pt}
%分岐を含むコレオグラフィのエンドポイント射影で重要なのは,ある参加者で発生した条件分岐をいかにして他の参加者に伝えるか,という仕組みを実現することである.これを実現するのが\textbf{マージ}である.
%PyChoralにおけるエンドポイント射影およびマージの理論はChoralを模倣している.
%if文におけるマージは,if文の条件式を判断した参加者以外が条件式の結果に相当する列挙型の値を受信して分岐することにより実現する.
%PyChoralではPython3.10から導入されたmatch文を使用している.
%match文のケースはCode \ref{ch}の7,10行目にある\texttt{select}メソッドで送信した列挙型の値で分ける.
%Code \ref{ch}の7行目から射影によりCode \ref{python}の6-8,11,12行目が生成される.
%Code \ref{ch}の10行目からは射影によりCode \ref{python}の6,9-12行目が生成される.
%\texttt{case \_} は共通のケースであるため,統合されて一つになる.
%\texttt{{\color{red}{Choice.HELLO}},{\color{blue}{Choice.BYE}}}は独立したケースであるため,マージ後のmatch文にそのまま加える.
%これにより,条件分岐に関係のない参加者も分岐を考慮した形でエンドポイント射影される.
%生成された\texttt{A,B}のPythonプログラムは,エンドポイント射影およびマージによりデッドロックが起こらずに連携して動作することが可能である.





\section{PyChoralの設計と実装}

PyChoralはエンドポイント射影において型情報を活用する.
これは,すべての式や文が射影される参加者と関連するかどうかに基づいて射影をするためである.
例えば,関数呼び出しの射影$\projection{f(\overline{e}):\tau}{A}$は引数と戻り値の型に,
射影される参加者の情報が含まれているかどうかで場合分けされる.%射影後の構文木が変化する.
ここで,$\projection{~}{A}$は参加者Aに対しての射影であることを示す.$\tau$は返値の型情報である.
引数$\overline{e}$は式のリストである.
%例えばメソッド呼び出し$\projection{\texttt{self.ChBA.com("bye"\mblue{@}B())}}{A}$
このようなエンドポイント射影を実現するために,PyChoralはPythonの基本型やクラスに参加者の情報を含める形で拡張している.
具体的な手法は次の2つである.%\textbf{型の拡張}と\textbf{式の拡張}で参加者の情報をPythonの型情報に組み込む形で拡張している.
\\
$\blacksquare$ \textbf{型の拡張}$~$参加者の情報を含む型として\texttt{At[T,R]}や\texttt{Choreograhy2[R1,R2]},\texttt{Choreograhy3[R1,R2,R3]}を定義する.
\texttt{At[T,R]}は参加者\texttt{R}に割り当てられる値の型\texttt{T}を表す型である.
PyChoralにおける\texttt{At[T,R]}はPythonの基本型\texttt{T}と同様に扱える.
\texttt{Choreograhy2[R1,R2]},\texttt{Choreograhy3[R1,R2,R3]}はPyChoralプログラムでクラス宣言をする際に継承する基底クラスの型である.
型パラメータに参加者の情報を持ち,参加者の人数で継承するクラスを指定する.
\\
$\blacksquare$ \textbf{式の拡張}$~$Pythonのすべての式が属する型であるオブジェクト型に\mblue{@}演算子を追加で定義する.
これにより,\texttt{exp\mblue{@}R()}のように式\texttt{exp}に参加者\texttt{R}を割り当てることができる.
ここで,\texttt{exp}の型が\texttt{T}であれば\texttt{exp\mblue{@}R()}の型は\texttt{At[T,R]}となる.
\\
\mblue{@}演算子および型クラス\texttt{At},\texttt{Choreograhy2},\texttt{Choreograhy3}はエンドポイント射影で参加者の情報を得るために用いられる.
これらを実体を持たないMypyの型宣言の形で定義することにより,各参加者のプログラムでは\mblue{@}演算子,型クラス\texttt{At,Choreograhy2,Choreograhy3}が消去される.%標準のPythonプログラムとして実行できる.





\section{アプリケーションによる評価}

Code \ref{merge}はChoralのプログラム例の一つであるParallel MergeSortをPyChoralで実装したものである(可読性向上のために型情報は省略する).
クラス\texttt{MergeSort}の参加者であるMergerと2つのSorterはパラメータ\texttt{M,S1,S2}として扱う.
2,3行目ではコンストラクタを記述している.\texttt{chS2M,chS1S2,chS2M}は型\texttt{T}の値を
双方向(\texttt{R1}$\rightleftarrows$\texttt{R2})に送り合えるチャネル\texttt{SymChannel[T,R1,R2]}の型をもつ.
4行目はメソッド\textsf{sort}の定義であり,リストを分割する必要性を確認する条件式で構成される.
5-15行目は分割する場合である.10行目でリストの中心(\texttt{pivot})を見つけ,それを境目にリストを分割して得られるサブリストを2つのSorterに送信する.
8,9行目では分割されたリストをさらにソートするために,MergerとSorterの役割を入れ替えながらMergeSortを再帰的に呼び出す.
%9行目では\texttt{S1}がMergerとなり,\texttt{S2,M}がSorterとなる.
%10行目では\texttt{S2}がMergerとなり,\texttt{M,S1}がSorterとなる.
このようにMergerとSorterの役割を代えながらリストの分割ができなくなるまでソートを行う.
各Sorterによって順序付けられたリストは15行目にあるメソッド\textsf{merge}の呼び出しによって結合される.
20行目にある\textsf{merge}はSorterがもつ2つのリストをMergerが結合するメソッドである.
このメソッドはローカルであり,リストの結合を逐次的なアルゴリズムのように実行する.
MergeSortのコードは,参加者の型情報により,リストのソートを並列して実行可能であるPythonプログラムを生成できる.
さらに,それらは理論的にデッドロックフリー性をもつ.
%MergeSortではまず,Mergerがリストを二分割し,2つのSorterに送信する.Sorterはそれぞれ受信したサブリストをソートし,その結果をMergerに返す.
%Mergerが順序付けられたサブリストを取得すると,それらをマージして完全に順序付けられたリストを返す.
%mb = Mergesort[S1,S2,M](self.s1, self.s2, self.m, self.chS1S2, self.chS2M, self.chMS1)
%mc = Mergesort[S2,M,S1](self.s2, self.m, self.s1, self.chS2M, self.chMS1, self.chS1S2)
      
\begin{lstlisting}[caption = MergeSort.py, label = merge]
class Mergesort(Choreograhy3[M,S1,S2]):
  def __init__(self,m,s1,s2,chMS1,chS1S2,chS2M):
    !#  constructor #!
  def sort(self,a):
    if len(a)@self.m() > 1@self.m():
      self.chMS1.select(MChoice.L@self.m())
      self.chS2M.select(MChoice.L@self.m())
      mb = Mergesort[S1,S2,M]( !# arguments #! )
      mc = Mergesort[S2,M,S1]( !# arguments #! )
      pivot = float(floor(len(a)/2))@self.m()
      b = a[0:int(pivot)]
      lhs = mb.sort(self.chMS1.com(b@self.m()))
      c = a[int(pivot):len(a)]
      rhs = mc.sort(self.chS2M.com(c@self.m()))
      return self.merge(self.chMS1.com(lhs),self.chMS1.com(rhs))
    else:
      self.chMS1.select(MChoice.R@self.m())
      self.chS2M.select(MChoice.R@self.m())
      return a
  def merge(self,lhs,rhs):
      ... 
\end{lstlisting}
%PyChoralを用いたアプリケーションとしてMergeSortを実装し,PyChoralの有用性を確認した。
%%%%%%%%%%%%%%%%%%%%%%%%%%%%%%%%%
\section{まとめ}

本研究はPythonを拡張したコレオグラフィックプログラミング言語PyChoralを構築した.
エンドポイント射影の際に参加者の情報を含む型情報を活用することで,不必要な式や文が排除された
各参加者のPythonプログラムが生成できる.
PyChoralをIoTや機械学習を扱うPythonプログラムに応用することが今後の課題である.

%本研究はPythonを拡張したコレオグラフィックプログラミング言語PyChoralを構築した.
%PyChoralプログラムからエンドポイント射影によって各参加者のPythonプログラムが生成される.
%これらは並行性に起因するエラーが生じないことが保証されている.これにより,
%複数参加者の連携動作を単一の言語でプログラミングが可能となり,Pythonプログラマの手助けとなる.
%PyChoralをIoTや機械学習を扱うPythonプログラムに応用していくことが今後の課題である.

%%%%%%%%%%%%%%%%%%%%%%%%%%%%%%%%%
%参考文献
\bibliography{reference_resume.bib}

\end{document}